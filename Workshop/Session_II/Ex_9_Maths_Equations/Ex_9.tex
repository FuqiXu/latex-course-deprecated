% Exercise 9: Here you will learn about using equations, matrices and inline mathmode.

\documentclass{article}

% 1. Note the use of amsmath package to have equations/mathmode enabled
\usepackage{amsmath}
\title{Equations}
\author{Krishna}
\date{}
\begin{document}
\maketitle

\section*{Inline Equation}

% 2. To create an inline equation use $equation_in_here$  Lets now type the famous energy-momentum relation $E^2 = m_0^2 c^4 + (pc)^2$

% *. an underscore x_{y} will create y as a subscript to x and x^y will generate y as a superscript.
The most famous equation in the world:  - The \textbf{energy-mass-momentum} relation as an in-line equation.

\section{equation and align environment}

% 3. Use the equation environment \begin{equation} \end{equation}to type your equations ;) Try to type the equation shown in Slide: 14-15 in Intro Presentation % % % %: \begin{equation} f(x)= \sin^2x+\frac{\tan \mathit{x}}{\log \mathit{x}} + \mathbf{X}^T\times\mathbf{X} \end{equation}


% 4. Now Lets ask LaTeX not to number the equation by using * at the end of the environment name \begin{equation*} \end{equation*}: % % % %:  \begin{equation*} \iint_{0}^{\infty} f(x,y)dx dy \end{equation*}

% 5. It's time to have multiple equations and align them. Note: & aligns the equations and see use of \nonumber. Uncomment the following section and see how it affects the output.
% then add & on either side of the equals sign on all equations and see the output like y & = & ax+ b

%\begin{eqnarray}
%	y    =  ax+b \nonumber\\
%	y+1  = ax+(b+1)\\
%	     = ax+(b+2)-1
%\end{eqnarray}

% 6. If you have a long equation you can fold the equation using a newline \\
%\begin{multline}
%f(x)=a1x_1+a2x_2+a3x_3+a_4x_4+
%\sqrt{a1x_1+a2x_2+a3x_3+a_4x_4}+\\
%a1x_1+a2x_2+a3x_3+a_4x_4+
%a1x_1+a2x_2+a3x_3+a_4x_4
%\end{multline}


% 7. A Matrix can be generated using pmatrix or bmatrix or vmatrix environment. Try to change pmatrix to bmatrix or vmatrix and see what difference it makes.
%\begin{equation}
%A_{m,n} =
% \begin{pmatrix}
%  a_{1,1} & a_{1,2} & \cdots & a_{1,n} \\
%  a_{2,1} & a_{2,2} & \cdots & a_{2,n} \\
%  \vdots  & \vdots  & \ddots & \vdots  \\
%  a_{m,1} & a_{m,2} & \cdots & a_{m,n}
% \end{pmatrix}
%\end{equation}

\end{document}