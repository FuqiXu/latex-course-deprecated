% Exercise 10: Formatting Tables

\documentclass{article}
\usepackage{multirow}
\usepackage{cleveref}
\begin{document}

\section{Table without borders}

% 1. use a tabular environment \begin{tabular}{cols} content... \end{tabular}. {cols} specify the alignment of each column in the table should be l-left c-center or r-right. 
% The delimiter for each column is & and a new line can be created using \\

\section{Table with borders}

% 2. To add vertical lines add | to {l c r} ==> {|l|c|r|} and use \hline for horizontal line between each rows. You can add multiple \hline for double line

% 3. To refer to the table, we should add a label to the table. Enclose the tabular environment inside a table environment. Create a unique label inside the table environment.

% 4. You can add a caption to the table using \caption{title} inside the table environment. You can refer to the table using cleveref package using ~\cref{key}. Package cleveref distinguishes between table, equation and figures automatically.

\section{Table with different borders}

% 5. Uncomment to see how you can add extra borders


%\begin{tabular}{|r|l|}
%  \hline \hline
%  7C0 & hexadecimal \\\hline
%  3700 & octal \\ \cline{2-2}
%  11111000000 & binary \\  \hline \hline
%  1984 & decimal \\   \hline
%\end{tabular}

\section{Table without specifying the width of last column}

% 6. Change the last column from l to p{xcm} where x is up to you to make the text in the last column fold.


%\begin{center}
%    \begin{tabular}{ | l | l | l | l |}
%    \hline
%    Day & Min Temp & Max Temp & Summary \\ \hline
%    Monday & 11C & 22C & A clear day with lots of sunshine.
%    However, the strong breeze will bring down the temperatures. \\ \hline
%    Tuesday & 9C & 19C & Cloudy with rain, across many northern regions. Clear spells
%    across most of Scotland and Northern Ireland,
%    but rain reaching the far northwest. \\ \hline
%    Wednesday & 10C & 21C & Rain will still linger for the morning.
%    Conditions will improve by early afternoon and continue
%    throughout the evening. \\    \hline
%    \end{tabular}
%\end{center}


\section{Multiple columns}

% 7. Uncomment the following section to see how to generate multiple columns
%\begin{tabular}{l|*{6}{c}|r|}
%Team              & P & W & D & L & F  & A & Pts \\
%\hline
%Manchester United & 6 & 4 & 0 & 2 & 10 & 5 & 12  \\
%Celtic            & 6 & 3 & 0 & 3 &  8 & 9 &  9  \\
%Benfica           & 6 & 2 & 1 & 3 &  7 & 8 &  7  \\
%FC Copenhagen     & 6 & 2 & 1 & 2 &  5 & 8 &  7  \\
%\end{tabular}


\section{Multi-column table}

% 8. To generate multicolumn (spread over two columns) use \multicolumn{cols}{pos}{text} cols = number of columns, pos = alignment.
\begin{tabular}{|l|l|l|}
  \hline
%  S.No &  \multicolumn{2}{|c|}{Team sheet} \\
  \hline
  1 & GK & Paul Robinson \\
  2 & LB & Lucus Radebe \\
  3 & DC & Michael Duberry \\
  4 & DC & Dominic Matteo \\
  5 & RB & Didier Domi \\
  6 & MC & David Batty \\
  7 & MC & Eirik Bakke \\
  8 & MC & Jody Morris \\
  9 & FW & Jamie McMaster \\
  10 & ST & Alan Smith \\
  11 & ST & Mark Viduka \\
  \hline
\end{tabular}

\section{Multi-Row}

9. Multi-row table uncomment to see how to format a multi-row table.
%\begin{tabular}{|l|l|l|}
%\hline
%\multicolumn{3}{|c|}{Team sheet} \\
%\hline
%Goalkeeper & GK & Paul Robinson \\ \hline
%\multirow{4}{*}{Defenders} & LB & Lucus Radebe \\
% & DC & Michael Duberry \\
% & DC & Dominic Matteo \\
% & RB & Didier Domi \\ \hline
%\multirow{3}{*}{Midfielders} & MC & David Batty \\
% & MC & Eirik Bakke \\
% & MC & Jody Morris \\ \hline
%Forward & FW & Jamie McMaster \\ \hline
%\multirow{2}{*}{Strikers} & ST & Alan Smith \\
% & ST & Mark Viduka \\
%\hline
%\end{tabular}
\end{document}