% Exercise 4: Creating Lists and Footnotes

% 1. Create a document class article and begin your document \documentclass[options]{style} \begin{document}
% 2. Create an enumeration of items in Ex_5.txt. To enumerate use begin evironment \begin{enumerate} and make sure you close it with a \end{enumerate}. To list the collection use \item before each collection.
% 3. To create a sub collection start another enumerate environment inside the outer \begin{enumerate} environment and continue listing using \item
% 4. Now, create a numbered list of collection. To do that use \begin{itemize} environment and make sure you close it with a \end{itemize}. It works similar to enumerate and you can list using \item.
% 5. Now try creating a description list using description environment. 

% 6. Start a new page using \newpage or \clearpage you can also try doing \cleardoublepage

% 7. Lets now past the article from Ex_5.txt. You can try typesetting characters/words as bold or italics using \textbf{text} and \textit{text}
% 8. Now create a margin note using \marginpar{text} and insert it where it says \[INSERT MARGIN NOTE HERE\]
% 9. Create a footnote using \footnote{text} and insert it where it says \[INSERT FOOTNOTE HERE\]

% 10. \end{document}


